\documentclass{article}
\usepackage{graphicx} % Required for inserting images
\graphicspath{ {./Images/} }
\usepackage{amsmath}
\renewcommand{\familydefault}{\sfdefault}

\title{Synthèse LINFO1252}
\author{Quentin Bodart}
\date{Q1 2024-2025}

\begin{document}
\maketitle
\tableofcontents
\pagebreak

\section*{Objectifs du cours}
    \begin{itemize}
        \item utiliser et comprendre les systèmes informatiques (i.p. GNU/Linux)
        \item utiliser les services fournis par les SE (systèmes d'exploitation)
        \item design et mise en oeuvre d'un SE
    \end{itemize}

\section{Le système informatique et le rôle du système d’exploitation}
\textbf{Syllabus} : https://sites.uclouvain.be/SystInfo/notes/Theorie/intro.html

    \subsection{Fondamentaux}
        \begin{itemize}
            \item Composants :
            \begin{itemize}
                \item CPU / Processeur
                \item Mémoire Principale (RAM)
                \item Dispositifs d’entrée/sortie (y.c. de stockage)
            \end{itemize}
            \item Fonctionnement d'un CPU
            \begin{itemize}
                \item Lire / écrire en mémoire vers / depuis des registres
                \item Opérations (calculs, comparaisons) sur ces registres
            \end{itemize}
            \item Jeux d'instructions
            \begin{itemize}
                \item x86\_64 (PC, anciens Mac)
                \item ARM A64 (Raspberry PI, iPhone, nouveaux Mac M1–3)
            \end{itemize}
        \end{itemize}
    
    \subsection{Architecture de von Neumann}
        \includegraphics[scale = .5]{von_neumann.png}
    
    \subsection{Fonctionnement d'un système informatique}
        \begin{itemize}
            \item La représentation des données se fait en \textbf{binaire}.
            \item Les opérations d'entrée-sortie se déroule de manière concurrente.
            \item Il y existe des contrôleurs distinct controlant chacun un type de périphérique.
            \item Chaque contrôleur possède une mémoire dédiée (buffer)
            \item Le processeur doit déplacer des donnés depuis/vers la
            mémoire principale depuis/vers ces buffers dédiés
            \item Le processeur suit un "fil" continu d’instructions
            \item Le contrôleur de périphérique annonce au
            processeur la fin d’une opération d’entrée/sortie en
            générant une interruption (signal électrique à destination du processeur)
        \end{itemize}

    \subsection{Traitement d'une interruption}
        Le processeur interrompt le fil d’exécution d’instructions courant et
        transfert le contrôle du processeur à une routine de traitement.
        Cette même routine détermine la source de l'interruption, puis restaure l'état du processeur
        et reprend le processus :\\
        \includegraphics[scale=.75]{interruption.png}

    \subsection{Accès direct à la mémoire}
        Direct Memory Access (DMA) désigne le fait de ne pas faire une interruption à chaque octet lu depuis un disque dur.
        Une interruption est tout de même faite à la fin du transfert d'un \textbf{bloc}.
    
    \subsection{Système informatique complet}
        \includegraphics*[scale=.85]{systeme_complet.png}
    
    \subsection{Rôle du système d'exploitation}
        Programmer directement au-dessus du matériel, gérer les interruption, etc... serait une trop grosse tâche pour le programmeur.\\\\
        3 rôles principaux
        \begin{itemize}
            \item Rendre l’utilisation et le développement d’applications
            plus simple et plus universel (portable d’une machine à
            une autre)
            \item Permettre une utilisation plus efficace des ressources
            \item Assurer l’intégrité des données et des programmes
            entre eux (e.g., un programme crash mais pas le système)
        \end{itemize}
    
    \subsection{Virtualisation}
        Le système d'expoitation \textbf{virtualise} les ressources matérielles afin de fonctionner de la même manière sur des systèmes avec des ressources et composants fort différents.
        Chaque SE doit trouver un compromis entre abstraction et efficacité !

    \subsection{Séparation entre mécanisme et politique}
        \begin{itemize}
            \item Un mécanisme permet le partage de temps
            \item Une politique arbitre entre les processus pouvant s’exécuter
            et le(s) processeur(s) disponibles
        \end{itemize}
        On peut définir des politiques d’ordonnancement différentes selon les contextes, mais sur la base du même mécanisme.
    
    \subsection{Modes d'exécution}
        \begin{itemize}
            \item mode utilisateur : programme utilisant les abstractions
            fournies par le SE ; certaines instructions sont interdites, comme par exemple:
            \begin{itemize}
                \item Accès à une zone mémoire non-autorisée (SegFault)
                \item De manière générale, toutes les instructions permettant de
                changer la configuration matérielle du système, comme la
                configuration ou la désactivation des interruptions
            \end{itemize}
            \item mode protégé : utilisé par le noyau du SE, toutes les
            instructions sont autorisées
            \item L’utilisation de fonctionnalités du SE par un processus
            utilisateur nécessite de passer d’un mode à l’autre : \textbf{appel système}
        \end{itemize}
    
    \subsection{Appel système}
        \begin{itemize}
            \item Un appel système permet à un processus utilisateur
            d’invoquer une fonctionnalité du SE
            \item Le processeur interrompt le processus, passe en mode
            protégé, et branche vers le point d’entrée unique du noyau :\\
            \includegraphics[scale=.75]{appel_systeme.png}
        \end{itemize}
            
        %TODO

\section{Utilisation de la ligne de commande}
\textbf{Syllabus} : https://sites.uclouvain.be/SystInfo/notes/
Theorie/shell/shell.html

        \subsection{Utilitaires UNIX}
            La philosophie lors de la création des utilitaires UNIX était
            de créer des outils les plus simples possible,
            donc d'avoir une seule tâche par outil :\\
            \includegraphics[scale=.5]{utilitaires_unix.png}
            Afin d'en savoir plus sur unecommande, il suffit d'utiliser l'utilitaire `man`.
        
        \subsection{Shell / Interpréteur de commandes}
            Rend possible l'interaction avec le SE.
            Il en existe plusieurs, mais le principal est `bash`.
            Il est toujours complémentaire à une \textbf{interface graphique}.
        
        \subsection{Flux et redirections}
            \includegraphics[scale=.5]{flux_std_et_redirections.png}\\
            Exemple de redirections :\\
            \includegraphics[scale=.5]{redirection_example.png}
        
            \subsection{Scripts}
                Un système UNIX peut exécuter du language machine ou des \textbf{languages interprétés}
                Un script commence par convention par les symboles $\#!$, référant à l'interpréteur, ici bin/bash :\\
                \includegraphics[scale=.7]{bash_script.png}\\\\
                Ils peuvent aussi contenir des \textbf{variables}:\\
                \includegraphics[scale=.5]{script_variables.png}\\\\
                et des \textbf{conditionnelles} :\\
                \includegraphics[scale=.5]{conditionnelles.png}\\
                et des \textbf{boucles for}:\\
                \includegraphics[scale=.6]{script_boucle_for.png}

    
\end{document}